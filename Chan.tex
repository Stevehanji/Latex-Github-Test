\documentclass[12pt,a4paper]{article}
\usepackage[utf8]{vietnam}
\usepackage[left=2cm, right=2cm, top=2cm, bottom=2cm]{geometry}
\usepackage{graphicx}
\usepackage{mathtools}
\usepackage{amssymb}
\usepackage{amsthm}
\usepackage{thmtools}
\usepackage{xcolor}
\usepackage{nameref}
\usepackage{hyperref}
\usepackage{fancyhdr}

% Title
\title{
    \LARGE Trường Đại học Tôn Đức Thắng\\
    \Large Khoa Toán - Thống Kê\\[2em]

    \LARGE TIỂU LUẬN GIỮA KỲ\\
    \Large Năm học: 2026 Học kỳ: 2\\
    Môn học: Soạn thảo tài liệu khoa học với Latex\\
    Mã Môn Học: C02045\\[2em]

    
    \LARGE Họ và tên sinh viên thực hiện: Trương Cảnh Chân\\
    \Large MSSV: C2500010\\

}
\date{\LARGE TP.HCM, ngày 4 tháng 2 năm 2026}

% new Theorem


\begin{document}

% --------- Trang 1 ----------------
\maketitle
\newpage

% --------- Trang 2 ----------------
Lời Cam kết:
\begin{itemize}
    \item[-] Tiểu luận này được tôi biên soạn lại bằng phần mềm \LaTeX với mục đích hoàn thành môn học thực hhành \textit{Soạn thảo tài liệu khoa học bằng \LaTeX}, ngoài ra không còn mục đích nào khác.
    \item[-] Tiểu luận này được biên soạn bằng \LaTeX dựa trên tài liệu sau: 
    \begin{itemize}
        \item[+] Tiêu đề:
        \item[+] Tác giả:
    \end{itemize}
\end{itemize}
\newpage

% --------- Trang 3 ----------------
\section{Introduction}
Let $(X,d)$ be a metric space. A mapping $T$ : $X \rightarrow X$ is said to be a $\varphi$-weak contraction if there exists a map $\varphi$ : $[0, \inf)$ with $\varphi(0) = 0$ and $\varphi(t) > 0$ all $t > 0$ such that

\begin{equation}
d(Tx, Ty) \leqslant d(x,y) - \varphi(d(x,y))
\end{equation}
for all $x, y \in X$.

Also two mappings $T, S$ : $X \rightarrow X$ are called generalized $\varphi$-weak contractions if there exits a map $\varphi$ : $[0, \inf) \rightarrow [0,\inf)$

\begin{equation}
d(Tx, Sy) \leqslant N(x,y) - \varphi(N(x,y))
\end{equation}
for all $x,y \in X$, where
\begin{equation}
    N(x, y) := \max\left\{d(x,y), d(x,Tx), d(y,Sy), \frac{1}{2}\bigl[d(x,Sy) + d(y, Tx)\bigr]\right\}
\end{equation}
The concept of the $\varphi$-weak contraction was defined by Alber and Guerre-Delabriere in 1997

% --------- Trang 4 ----------------


% --------- Trang 5 ----------------


% --------- Trang 6 ----------------


% --------- Trang 7 ----------------


% --------- Trang 8 ----------------
\newpage
\renewcommand{\refname}{References}
\begin{thebibliography}{99}
    \bibitem{TaiLieu1} Ya.I. Alber, S. Guerre-Delabriere, Principles of weakly contractive maps in Hilbert spaces, in: I. Gohberg, Yu. Lyubich (Eds.), New Results in Operator Theory, in: Advances and Appl., vol. 98, Birkhäuser, Basel, 1997, pp. 7–22.
    \bibitem{TaiLieu2} Q. Zhang, Y. Song, Fixed point theory for generalized $\varphi$-weak contractions, Appl. Math. Lett. 22 (2009) 75–78.
    \bibitem{TaiLieu3} B.E. Rhoades, Some theorems on weakly contractive maps, Nonlinear Anal. 47 (2001) 2683–2693.
    \bibitem{TaiLieu4} S. Banach, Sur les opérations dans les ensembles abstraits et leur application aux équations intégrales, Fund. Math. 3 (1922) 133–181 (in French).
    \bibitem{TaiLieu5} D.W. Boyd, J.S.W. Wong, On nonlinear contractions, Proc. Amer. Math. Soc. 20 (1969) 458–464.
    \bibitem{TaiLieu6} S. Reich, Some fixed point problems, Atti. Accad. Naz. Lincei, Rend. Cl. Sci. Fis. Mat. Nat. (8) 57 (3–4) (1974) 194–198. (1975).
    \bibitem{TaiLieu7} B.DjafariRouhani,SirousMoradi,Commonfixedpointofgeneralized $\varphi$-weakcontractivemulti-valuedandsinglevaluedmappings,FixedPointTheory
Appl. (2010) doi:10.1155/2010/708984.
    \bibitem{TaiLieu8} G. Jungck, B.E. Rhoades, Fixed point for set valued functions without continuity, Indian J. Pure Appl. Math. 29 (3) (1998) 227–238.
    \bibitem{TaiLieu9} A. Djoudi, F. Merghadi, Commonfixedpoint theoremsfor mapsunderacontractive condition of integral type, J. Math. Anal. Appl. 341 (2008) 953–960.
\end{thebibliography}


\end{document}