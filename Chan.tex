\documentclass[12pt,a4paper]{article}
\usepackage[utf8]{vietnam}
\usepackage[left=2cm, right=2cm, top=2cm, bottom=2cm]{geometry}
\usepackage{graphicx}
\usepackage{mathtools}
\usepackage{amssymb}
\usepackage{amsthm}
\usepackage{thmtools}
\usepackage{xcolor}
\usepackage{nameref}
\usepackage{hyperref}
\usepackage{fancyhdr}
\usepackage{indentfirst}

% amsthm
\numberwithin{equation}{section}

% Title
\title{
    \LARGE Trường Đại học Tôn Đức Thắng\\
    \Large Khoa Toán - Thống Kê\\[2em]

    \LARGE TIỂU LUẬN GIỮA KỲ\\
    \Large Năm học: 2026 Học kỳ: 2\\
    Môn học: Soạn thảo tài liệu khoa học với Latex\\
    Mã Môn Học: C02045\\[2em]

    
    \LARGE Họ và tên sinh viên thực hiện: Trương Cảnh Chân\\
    \Large MSSV: C2500010\\

}
\author{}
\date{\LARGE TP.HCM, ngày 4 tháng 2 năm 2026}

% new Theorem
\newtheorem{theorem}{Theorem}[section]
\newtheorem{definition}{Definition}[section]

\begin{document}

% --------- Trang 1 ----------------
\maketitle
\newpage

% --------- Trang 2 ----------------
Lời Cam kết:
\begin{itemize}
    \item[-] Tiểu luận này được tôi biên soạn lại bằng phần mềm \LaTeX với mục đích hoàn thành môn học thực hhành \textit{Soạn thảo tài liệu khoa học bằng \LaTeX}, ngoài ra không còn mục đích nào khác.
    \item[-] Tiểu luận này được biên soạn bằng \LaTeX dựa trên tài liệu sau: 
    \begin{itemize}
        \item[+] Tiêu đề:
        \item[+] Tác giả:
    \end{itemize}
\end{itemize}
\newpage

% --------- 1 Introduction ----------------
\section{Introduction}
Let $(X,d)$ be a metric space. A mapping $T$ : $X \rightarrow X$ is said to be a $\varphi$-weak contraction if there exists a map $\varphi$ : $[0, \infty)$ with $\varphi(0) = 0$ and $\varphi(t) > 0$ all $t > 0$ such that
% ----- Equation 1-1 ----
\begin{equation} \label{eq1-1}
d(Tx, Ty) \leqslant d(x,y) - \varphi(d(x,y))
\end{equation}
for all $x, y \in X$.

Also two mappings $T, S$ : $X \rightarrow X$ are called generalized $\varphi$-weak contractions if there exits a map $\varphi$ : $[0, \infty) \rightarrow [0,\infty)$
% ----- Equation 1-2 ----
\begin{equation} \label{eq1-2}
    d(Tx, Sy) \leqslant N(x,y) - \varphi(N(x,y))
\end{equation}
for all $x,y \in X$, where
% ----- Equation 1-3 ----
\begin{equation} \label{eq1-3}
    N(x, y) := \max\left\{d(x,y),\, d(x,Tx), d(y,Sy),\, \frac{1}{2}\bigl[d(x,Sy) + d(y, Tx)\bigr]\right\}
\end{equation}
The concept of the $\varphi$-weak contraction was defined by Alber and Guerre-Delabriere \cite{TaiLieu1} in 1997, and the generalized $\varphi$-weak
contraction was defined by Zhang and Song \cite{TaiLieu2} in 2009. Rhoades \cite[Theorem~2]{TaiLieu3} proved the following fixed point theorem 
for $\varphi$-weak contraction single-valued mappings, giving another generalization of the Banach contraction principle \cite{TaiLieu4}

% ----- Theorem 1.1 -----
\begin{theorem} \label{Theorem1-1}
    Let $(X,d)$ be a complete metric space, and let $T$ : $X \rightarrow X$ be a mapping such that
    % ----- Equation 1-4 ----
    \begin{equation} \label{eq1-4}
        d(Tx, Sy) \leqslant d(x,y) - \varphi(d(x,y))
    \end{equation}
    for every $x,y \in X$ (i.e. it is $\varphi$-weakly contractive), where $\varphi$ : $[0, +\infty) \rightarrow [0, +\infty)$ is a continuous and nondecreasing function
    with $\varphi(0) = 0$ and $\varphi(t) > 0$ for all $t > 0$.Then $T$ has a unique fixed point.
\end{theorem}

On choosing $\psi(t) = t - \varphi(t)$, $\varphi$-weak contractions become mappings of Boyd and Wong type \cite{TaiLieu5}, and on defining
$k(t) = \frac{1 - \varphi(t)}{t}$ for $t > 0$ and $k(0) = 0$, then $\varphi$-weak contractions become mappings of Reich type \cite{TaiLieu6}.

In 2009 Zhang and Song \cite{TaiLieu2} proved the following theoremon the existence of acommon fixed point for two single-valued
generalized $\varphi$-weak contraction mappings.

% ----- Theorem 1.2 -----
\begin{theorem}\label{Theorem1-2}
    Let $(X,d)$ be a complete metric space, and let $T$, $S$ : $X \rightarrow X$ be two mappings such that for all $x,y \in X$
    % ----- Equation 1-4 ----
    \begin{equation} \label{eq1-5}
        d(Tx, Sy) \leqslant N(x,y) - \varphi(N(x,y))
    \end{equation}
    (i.e. they are generalized $\varphi$-weakcontractions), where $\varphi$ : $[0,+\infty) \rightarrow [0,+\infty)$ is a l.s.c. function with $\varphi(0) = 0$ and $\varphi(t) > 0$
    for all $t > 0$. Then there exists a unique point $x \in X$ such that $x = Tx = Sx$.
\end{theorem}

{
    \small Recently Rouhani and Moradi \cite{TaiLieu7} extended Theorems \ref{Theorem1-1} and \ref{eq1-2} to multivalued mappings.
    
    In Section 3, we extend Theorem \ref{Theorem1-1} by assuming $\varphi$ to be only l.s.c., and extend Theorem \ref{Theorem1-2}
}

\newpage
% --------- 2 Preliminaries ----------------
\section{Preliminaries}
In this work, $(X,d)$ denotes a complete metric space and $E$ denotes a nonempty closed subset of $X$.

\begin{definition}
    (See \cite{TaiLieu8}). Let $f,g$ be two self-mappings of a metric space $(X,d)$. $f$ and $g$ are said to be weakly compatible if
    for all $t \in X$ the equality $ft = gt$ implies $fgt = gft$.
\end{definition}

\begin{definition}
    Two mappings $T$, $S$ : $E \rightarrow E$ are called generalized $\varphi_f$-weakly contractive if there exist two maps
    $\varphi$ : $[0,\infty) \rightarrow [0,\infty)$ and $f$ : $E \rightarrow X$ with $\varphi(0) = 0$ and $\varphi(t) > 0$ forall $t > 0$ such that
    \begin{equation} \label{eq2-1}
        d(Tx, Sy) \leqslant M(x,y) - \varphi(M(x,y))
    \end{equation}
    for all $x, y \in X$, where
    \begin{equation} \label{eq2-2}
        M(x, y) := \max\left\{d(fx,fy),\, d(fx,Tx), d(fy,Sy),\, \frac{1}{2}\bigl[d(fx,Sy) + d(fy, Tx)\bigr]\right\}
    \end{equation}
\end{definition}


% --------- 3 Main results ----------------
\section{Main results}


% --------- 4 Conclusion and future directions ----------------
\section{Conclusion and future directions}


% --------- 5 Acknowledgements ----------------
\section*{Acknowledgements}
The authors sincerely thank the referees for their careful reading of this work and valuable suggestions.


% --------- 6 References ----------------
\newpage
\renewcommand{\refname}{References}
\begin{thebibliography}{99}
    \bibitem{TaiLieu1} Ya.I. Alber, S. Guerre-Delabriere, Principles of weakly contractive maps in Hilbert spaces, in: I. Gohberg, Yu. Lyubich (Eds.), New Results in Operator Theory, in: Advances and Appl., vol. 98, Birkhäuser, Basel, 1997, pp. 7–22.
    \bibitem{TaiLieu2} Q. Zhang, Y. Song, Fixed point theory for generalized $\varphi$-weak contractions, Appl. Math. Lett. 22 (2009) 75–78.
    \bibitem{TaiLieu3} B.E. Rhoades, Some theorems on weakly contractive maps, Nonlinear Anal. 47 (2001) 2683–2693.
    \bibitem{TaiLieu4} S. Banach, Sur les opérations dans les ensembles abstraits et leur application aux équations intégrales, Fund. Math. 3 (1922) 133–181 (in French).
    \bibitem{TaiLieu5} D.W. Boyd, J.S.W. Wong, On nonlinear contractions, Proc. Amer. Math. Soc. 20 (1969) 458–464.
    \bibitem{TaiLieu6} S. Reich, Some fixed point problems, Atti. Accad. Naz. Lincei, Rend. Cl. Sci. Fis. Mat. Nat. (8) 57 (3–4) (1974) 194–198. (1975).
    \bibitem{TaiLieu7} B.DjafariRouhani,SirousMoradi,Commonfixedpointofgeneralized $\varphi$-weakcontractivemulti-valuedandsinglevaluedmappings,FixedPointTheory
Appl. (2010) doi:10.1155/2010/708984.
    \bibitem{TaiLieu8} G. Jungck, B.E. Rhoades, Fixed point for set valued functions without continuity, Indian J. Pure Appl. Math. 29 (3) (1998) 227–238.
    \bibitem{TaiLieu9} A. Djoudi, F. Merghadi, Commonfixedpoint theoremsfor mapsunderacontractive condition of integral type, J. Math. Anal. Appl. 341 (2008) 953–960.
\end{thebibliography}


\end{document}